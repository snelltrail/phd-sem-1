\documentclass{article}[11]

\usepackage{bm, hyperref,amssymb}
\newcommand{\RR}{\mathbb{R}}

\title{What is geometry?}
\author{Daniel Snell}
\date{}

\begin{document}

\maketitle

\vspace{-2.5em}
\begin{center}
	Supervisor: A. Rod Gover
\end{center}
%\vspace{1em}

Geometry is one of the oldest areas of mathematics.
Mathematicians throughout history from many different cultures and schools have contributed their ideas and understanding to the field, all attempting to answer the question posed by the title of this talk.
This seminar will focus on several ideas from the last two centuries which seek to generalize $\left(\RR^n, \langle \cdot , \cdot \rangle\right)$, i.e. Euclidean space with its inner product structure.

In the first half of the talk, we will see a progression of ideas from simple Euclidean geometry, through the two different generalizations provided by Riemann and Klein geometries, and culminating in Cartan's sophisticated and powerful \emph{espaces g\'{e}n\'{e}ralis\'{e}s}, which these days go by the eponymous \emph{Cartan geometries} (these developments are well described in the book~\cite{Sharpe}).

In the latter half, we will discuss some modern examples of Cartan geometry of significance to my own research, specifically two \emph{parabolic geometries}~\cite{CS-book}: conformal and projective geometry.
The \emph{tractor bundles} of Bailey, Eastwood and Gover~\cite{BEG} will be introduced and we will see how they provide a good framework for asking and answering questions about such structures in an invariant way.
Finally, we will see this in practice in the context of distinguished curves for conformal and projective geometry~\cite{GST}.
\bibliographystyle{ieeetr}

\bibliography{../GST}

\end{document}
